\section{Tema}%
ANÁLISE DE FLUXO DE DISCIPLINAS NA GRADE ACADÊMICA DA UNB: seus reflexos na trajetória dos discentes.

\section{Problema}%
A necessidade de avaliar as políticas públicas, com destaque para as que são voltadas para a educação superior. Surge, assim, um anseio de conhecer os resultados das políticas de assistência social especificamente a de auxílio alimentação. A busca desses resultados deve ser contínua e sempre atualizada, deve também considerar a influência de características socioeconômicas e a eficácia do instrumento, com o objetivo de tornar as políticas mais eficientes.

\section{Objetivos}%

\subsection{Objetivo Geral}
O objetivo geral do estudo é, assim, analisar, a partir dos dados obtidos do Sistema de Informações Acadêmicas de Graduação (SIGRA) e do Sistema de controle de acesso ao Restaurante Universitários (SISRU), no período de outubro de 2014 a outubro de 2017, se a adesão ou não ao plano de Assistência Estudantil adotada pela Universidade de Brasília foi relevante para o aumento do rendimento acadêmico dos discentes. Os resultados dessa pesquisa devem promover a discussão acerca do assunto e promover uma profunda reflexão sobre tais políticas públicas.

\subsection{Objetivo específicos}
\begin{enumerate}
	\item Obter dados comparativas sobre a política de assistência estudantil alimentar;
	\item Avaliar a eficácia da política de assistência estudantil alimentar implementada pela Universidade de Brasília;
	\item Comparar beneficiários e não beneficiários de assistência estudantil alimentar;
	\item Avaliar o rendimento acadêmico de discentes beneficiados pela assistência ao decorrer do tempo na Universidade de Brasília; 
	\item Analisar rendimento acadêmico dos alunos no período de outubro de 2014 a outubro de 2017;
	\item Analisar a evasão dos discentes beneficiados pela assistência estudantil alimentar em comparação com os demais;
	\item Sugerir medidas que possibilitem melhor desempenho na implementação da política de assistência estudantil da universidade, no que diz respeito ao alcance dos objetivos a que ela se propõe, se for o caso.
\end{enumerate}%
\section{Citações}%
   Citações: \cite{Beatriz2007ABrasileiro}, \cite{Rita2006AEvasao}, 

