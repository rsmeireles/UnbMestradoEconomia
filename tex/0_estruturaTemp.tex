\section{Tema}%
ANÁLISE DE FLUXO DE DISCIPLINAS NA GRADE ACADÊMICA DA UNB: seus reflexos na trajetória dos discentes.

\section{Problema}%
A grade curricular do turno diurno é consideravelmente maior que a do noturno em

termos de disponibilidade de horários de aula. Uma situação crítica acontece recorrente-
mente em cursos do período noturno pois a grade fora construída de uma maneira que

impossibilita o aluno de cursar outras disciplinas caso ele não acompanhe o fluxo estimado,
ou seja, reprove ou não pegue uma disciplina. Essa situação faz com que o aluno percorra
um caminho a margem do fluxo inicialmente fornecido, caso não tenha disponibilidade de
cursar disciplinas no turno diurno, pois a oferta de disciplinas alternativas é limitada e a
grade do período noturno foi construída de forma tão justa que impossibilita o discente
de adiantar outras disciplinas de sua grade travando o fluxo e consequentemente forçando
o mesmo a não ter o que cursar na universidade. Diante do exposto, fica os seguintes

questionamentos: qual é o efeito de uma estrutura mal concebida de curso sobre o pro-
cesso de evasão na Universidade de Brasília? Nossa estrutura pode estar contribuindo

para esse processo? Em que medida os nossos algoritmos de alocação de matrícula são de
fato eficientes e contribuem para uma formatura em tempo razoável?

Diante do cenário exposto o presente projeto de pesquisa busca apresentar um diag-
nóstico e uma possível solução de otimização dos sistemas de distribuição de disciplinas

ofertadas pelos cursos de graduação da UnB nos turnos noturno e diurno visando mini-
mizar a retenção, a evasão e o tempo de formatura dos discentes na UnB.

\section{Objetivos}%

\subsection{Objetivo Geral}

O objetivo geral da pesquisa proposta é analisar os algoritmos de distribuição de
disciplinas na grade curricular dos cursos ofertados pela Universidade de Brasília (UnB)
nos seus diversos campi sob a ótica da evasão dos discentes.

\subsection{Objetivo específicos}

Os objetivos específicos são:
\begin{enumerate}
	\item Realizar um estudo sobre o algoritmo de distribuição de disciplinas e compreender
sua dinâmica;
	\item Estruturar o algoritmo de distribuição de disciplinas;
	\item Analisar o tempo estimado do fluxo de disciplinas dos cursos ofertados pela Universidade
de Brasília;
\item Analisar o tempo real do fluxo de disciplinas dos cursos ofertados pela Universidade
de Brasília;
\item Correlacionar o tempo do fluxo de disciplinas dos cursos ofertados com o processo
de evasão da Universidade de Brasília.
\end{enumerate}%
\section{Citações}%
   Citações: \cite{Beatriz2007ABrasileiro}, \cite{BRITO2013ImplementacaoEvasao}, 
\cite{Diogo2016PercepcoesPreventivas}, \cite{GlaucoPeresdaSilva2013AnaliseDeterminantes}, \cite{KozelskiPoliticas}, \cite{Mezzari2013EstrategiasEvasao}, \cite{Economia2016NEWTONUnB}, \cite{Sales2014MetodosProfissional}, \cite{SantosBaggi2011EvasaoBibliografica}, \cite{SantosJunior2017A1990}, \cite{Tontini2014Pode-seSuperior}, \cite{Vieira2013DepartamentoFinanciamento}
